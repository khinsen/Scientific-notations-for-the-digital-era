\section{The structure of scientific knowledge}
\label{scientific-knowledge}

For the following discussion of scientific notation, it is useful to classify scientific knowledge into three categories: factual, procedural, and conceptual knowledge. Factual knowledge consists of the kind of information one can record in tables, diagrams, or databases: the density of water, the names of the bones in the human body, the resolution of an instrument, etc. Procedural knowledge is about doing things, such as using a microscope or finding the integral of a function. Conceptual knowledge consists of principles, classifications, theories, and other means that people use to organize and reason about facts and actions.

Factual and procedural knowledge relies on conceptual knowledge. A table listing the density of water at different temperatures refers to the concepts of density and temperature. Instructions for using a microscope refer to concepts such as sample or focus. Interpreting factual or procedural knowledge requires a prior knowledge of the underlying conceptual knowledge.

Conceptual knowledge has a hierarchical structure, with the definition of every concept referring to more fundamental concepts. This leads to the question of where this recursive process ends, i.e. what the most fundamental concepts are. When considering human knowledge as a whole, this is a non-trivial problem in epistemology. For a discussion of scientific knowledge, and in particular for the present discussion of scientific notation, it is sufficient to consider the concepts of everyday life as a given base level.

Factual and procedural knowledge often refer to each other. The statement "The orbit of the Moon around the Earth is reproduced to precision A by solving Newton's equations for the solar system using numerical algorithm B and initial values C" is factual knowledge, once specific A, B, and C are provided. But algorithm B is procedural knowledge, which in turn refers to some other factual knowledge, such as the masses of the Sun and its planets.

A final missing piece is metadata. Every piece of factual and procedural knowledge comes with information attached to it that describes its provenance and known limits of validity. A table showing the density of water at different temperatures should state how, when, under which conditions, and by who the listed values were obtained. It should also provide an estimate of the values' accuracy.

In summary, the structure of scientific knowledge can be described as a web of factual and procedural knowledge items that refer to each other, and which are expressed in terms of concepts from the universe of conceptual knowledge. The latter consists of multiple layers, with concepts from everyday life at the bottom. Every layer refers to concepts from lower, more fundamental layers.