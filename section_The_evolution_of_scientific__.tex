\section{The evolution of scientific communication}
\label{evolution}

Most of scientific communication takes place through research articles, which are narratives that propose new factual and procedural knowledge, occasionally also new concepts, and try to convince the scientific community of the pertinence of this information. Over time, as a given subject area becomes better understood, the scientific community usually reaches a consensus about which concepts are the most useful for describing its phenomena. The knowledge from a large number of research articles is then distilled into review articles, monographs, and other reference works. The knowledge considered most fundamental ends up in textbooks for transmission to the next generation of scientists. Each unit of scientific communication is written for a specific audience, and relies on a stack of conceptual layers that this audience is expected to be familiar with.

Before the use of computers, scientific knowledge was mainly recorded on paper, using three forms of notation: written language, images, and tables. Written text combines plain language, domain-specific vocabulary, and shorthand notation such as mathematical formulas. Images include both drawings and observations captured in photographs, radiographs, etc. Tables represent datasets, which are most often numerical.

There is a close relation between the conceptual knowledge on which a narrative relies and the notation that it employs. Domain-specific vocabulary directly names relevant concepts. Shorthand notation replaces frequently used words and lengthy sentences that involve these concepts. For example, Newton's laws of motion are commonly written as

\[
   F = m \cdot a
\]

whose full-length equivalent is "The force acting on a point mass is equal to the product of its mass and its acceleration." Force, mass, and acceleration are concepts from mechanics and $F$, $m$, and $a$ are conventional shorthands for them. The symbols $=$ and $\cdot$ are shorthands for the concepts of equality and product, both of which come from more fundamental conceptual layers in mathematics.

The standardization of scientific notation is variable and usually related to the stability of the concepts that it expresses. Well-established conceptual layers come with a consensus notation, whereas young conceptual layers can be expressed very differently by different authors. Scientists "play around" with both the concepts and the notations in rapidly evolving fields, before eventually settling on a consensus that has proven to work well enough. Even the most basic aspects of mathematical notation that we take for granted today were at some time the subject of substantial tinkering \cite{_history_2016}. Moreover, even a consensus notation is not completely rigid. Personal and disciplinary tastes and preferences are one cause of variation. As an example, there are several common notations for distinguishing vectors from scalars in geometry. Another cause is the limited number of concise names and labels. For example, the preference for one-letter names in mathematical formulas, combined with the useful convention of each letter having only one meaning in a given document, often imposes deviations from consensus naming schemes.

This pattern of a high variability during innovation phases giving way to consensus as a field or technology matures is ubiquitous in science and engineering. The time scale of the consolidation process is often decisive for reaching a satisfactory consensus. The lack of consensus in mature technology is felt as a nuisance by its users. A good example is the pointless diversity in chargers for mobile phones. On the other hand, premature consensus creates badly adapted technology that is difficult to get rid of. Computing technology is particularly affected by this problem. In fact, most of the standardized technology in computing -- file formats, programming languages, file systems, Internet protocols, etc. -- is no longer adequate for today's requirements. The reason is that the technical possibilities -- and, as a consequence, user demands -- evolve too fast for an orderly consensus formation, whose time scale is defined by human cognitive processes and social interactions that, unlike technological progress, have not seen any spectacular acceleration.
