\section{Designing digital scientific notations}
\label{design-guidelines}

The main conclusion from the analysis that I have presented in this essay is that digital scientific notations should be based on formal languages with the following properties:

 - **Small and simple**: each formal language must be so small and simple that a scientist can memorize it easily and understand its semantics in detail.

 - **Flexible**: a scientist must be able to create modifications of existing languages used in his/her field in order to adapt them to new requirements and personal preferences.

 - **Interoperable**: composition of digital knowledge items expressed in different languages must be possible with reasonable effort.

The two examples (section~\ref{simple-and-flexible}) I have presented above suggest that a good approach is to define a framework of languages and implement generic tools for common manipulations. The foundation of this framework should provide basic data types and data structures:

 - numbers (integers, rationals, floating-point, machine-level integers)
 - symbols
 - text
 - N-dimensional arrays
 - trees
 - sets
 - key-value maps (also called associative arrays, hash tables, or dictionaries)

The representation of these fundamental data types in terms of bit sequences can be based on existing standards such as XML (text) or HDF5 (binary). It is probably inevitable to have multiple such representations to take into account conflicting requirements of different application domains. As long as automatic loss-less interconversion can be ensured, this should not be an obstacle to interoperability. An added advantage of keeping the lowest level of representation flexible is the possibility to adapt to future technological developments, for example IPFS \cite{benet_ipfs_2014} whose "permanent Web" approach seems well adapted to preserving the scientific record.

There should also be a way to represent algorithms, but it is less obvious how this should best be done. Any of the common Turing-complete formalisms (lambda calculus, term rewriting, ...) could be used, but it may turn out to be useful to have access to less powerful formalisms as well, because they facilitate the automated analysis of algorithms.

A next layer could introduce domain-specific but still widely used data abstractions, e.g. from geometry. For much of mathematics, the OpenMath content dictionaries \cite{openmath_society_openmath_2000} could be adopted. On top of this layer, each scientific community can build its own digital scientific notations, and each scientist can fine-tune them to specific needs.

An illustration of how these principles can be applied is given by the MOlecular SimulAtion Interchange Conventions (MOSAIC) \cite{hinsen_mosaic_2014}, which define a digital notation for molecular simulations. MOSAIC lacks the common layer of data types listed above, and is therefore not easily interoperable with other (future) digital notations. It does, however, define data structures specific to molecular simulations in terms of more generic data structures, in particular arrays. MOSAIC defines two bit-level representations, based on XML and HDF5. A Python library \cite{hinsen_pymosaic_2014} proposes three further implementations in terms of Python data structures, and implements I/O to and from the XML and HDF5 representations.

Traditional scientific notations have evolved as a byproduct of scientific research, and digital scientific notations will have to evolve in the same way in order to be well adapted to the task. In this spirit, the ideas listed in this section are merely the basis I intend to use in my own future work, but they may well turn out to be a dead end in the long run. I would like to encourage computational scientists to develop their own approaches if they think they can do better. As I have stated in the introduction, my goal with this essay is not to propose solutions, but to expose the problem. If computational scientists start to think about "digital scientific notation" rather than "file formats" and "programming languages", I consider my goal achieved.
