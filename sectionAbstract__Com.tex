\begin{abstract}
Computers have profoundly changed the way scientific research is done. Whereas the importance of computers as research tools is evident to everyone, the impact of the digital revolution on the representation of scientific knowledge is not yet widely recognized. An ever increasing part of today's scientific knowledge is expressed, published, and archived exclusively in the form of software and electronic datasets.  In this essay, I compare these digital scientific notations to the the traditional scientific notations that have been used for centuries, showing how the digital notations optimized for computerized processing are often an obstacle to scientific communication and to creative work by human scientists. I analyze the causes and propose guidelines for the design of more human-friendly digital scientific notations.
\end{abstract}